\section{Points to Ponder regarding RAC process}


I.	Need a Call for Proposals (hereafter, CALL). 

II.	Materials needed to make a call for proposals (CALL) :


1)	What resources are on offer?  Definitions and available volumes and units


2)	New proposal “form” tailored to 1) above


a.	Resources being requested


b.	Science to be done  and methods to be used


c.	Justification for requested resources


d.	Duration of resource privileges  (1 semester?)


e.	Continuation of prior resource grant?


f.	Student PhD thesis?



III.	Long term status?  Or continuation requests each semester?


1)	What balances should be in place, so as not to “mortgage the future”


a.	Enunciate and communicate principles in the CALL


2)	Accommodate answer in I.2) above



IV.	Mechanism to track resource usage by awarded programs


1)	Do certain resources need to be scheduled to avoid temporal overloads?

V.	Scoping the selection process:


1)	What volume of proposals to expect?  Will likely change with time


a.	Affects number of panels, etc.


2)	How often should proposals be evaluated (semester offset from TAC?)


3)	Should certain kinds of proposals be evaluated on a continuing basis? E.g. like the Gemini fast-turnaround proposals?

VI.	Staff support for proposers 


1)	This will be a new process – proposers will have lots of questions in the beginning proposal cycles – need to set up help desk with knowledgeable people



VII.	Will proposals need a technical review?  


1)	If yes, who will do it?

VIII.	How will process interact with time domain project needs?


1)	See also IV.3) above

IX.	NOIRLab TAC proposals are now anonymized 


1)	Probably too much to handle for early cycles of RAC, but evolve to it later


a.	Communicate why and future intent on CALL



\url{https://drive.google.com/drive/folders/1AXliGxpAR-Aa5ESL9kmjB7TO7RwE1GG1?usp=drive_link}

\url{https://docs.google.com/spreadsheets/d/1r6JH0_5ROdSZ7I9_N4eSEHGbYgOO2QOwW_70IGo8RSg/edit?usp=sharing}


      






